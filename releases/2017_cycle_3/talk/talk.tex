\documentclass{beamer}

\usepackage{amsmath}
\usepackage{amssymb}
\usepackage{listings}
\usepackage{pgffor}
%\usepackage[backend=biber,
%                style=authoryear,
%                natbib=true,
%                style=numeric-comp
%                ]{biblatex}
%\addbibresource[location=remote]{https://raw.githubusercontent.com/CJ-Wright/Masters_Thesis/master/thesis.bib}
%
%\AtBeginSection[]{
%  \begin{frame}
%  \vfill
%  \centering
%  \begin{beamercolorbox}[sep=8pt,center,shadow=true,rounded=true]{title}
%    \usebeamerfont{title}\insertsectionhead\par%
%  \end{beamercolorbox}
%  \vfill
%  \end{frame}
%}

\graphicspath{{../figs/si_cal/}{../figs/}}

\usetheme{Berkeley}

\title{XPD 2017 Cycle 3 Release}
\author{Christopher J. Wright
\and Timothy Liu
\and Simon J. L. Billinge \\
@CJ-Wright
\and @chiahaoliu
\and @sbillinge
}
\titlegraphic{\includegraphics[width=5cm, draft=false]{../../logos/294OfficialColumbiaLogo.eps}}

%\logo{%
%  \makebox[0.95\paperwidth]{%
%    \includegraphics[width=1cm,keepaspectratio]{../../../mega-logo}%
%    \hfill%
%    \includegraphics[width=3.5cm,keepaspectratio]{../logos/294OfficialColumbiaLogo.eps}%
%  }%
%}


\begin{document}
\frame{\titlepage}

\section{Introduction}
\begin{frame}
\frametitle{Features}
\begin{itemize}
\item General Refactor
\item Enhanced Metadata
\item Tramp/Tlist shutter protocol
\item Data Reduction and Visualization Pipeline
\end{itemize}
\end{frame}

\section{General Refactor}
\begin{frame}
\frametitle{General Refactor}
Many pieces of the XPD stack needed to be refactored due to the changes in
Bluesky and Databroker, including:
\begin{itemize}
\item Removal of ``dot'' access from headers
\item Databroker named configurations
\item Removal of dereferencing of documents
\end{itemize}
\end{frame}

\section{Enhanced Metadata}
\begin{frame}
\frametitle{Enhanced Metadata - Motivation}
\begin{itemize}
\item Metadata is very important for searching data, mineing data, and using data
effectivly.

\item The XPD stack is designed to make the additon of metadata as easy as possible,
enabling users to insert sample data via a spreadsheet and tacking metadata
about the experiment (eg calibration data) automaticaly.

\item Two major features were implemented:
\begin{itemize}
\item Long Term Beamline Configuration
\item Server Client keys
\end{itemize}
\end{itemize}
\end{frame}

\begin{frame}
\frametitle{Long Term Beamline Configuration}
\begin{itemize}
\item There is a dedicated file
\texttt{.../xpdConfig/XPD\_beamline\_config.yml}\\
in yaml format to hold this kind of information.

\item Please feel free to put whatever information you deam nessisary in this file.

\item This information will be very helpful to make sure that when the data is
queried one can make certain that the beamline configurations are comparable.
\end{itemize}
\end{frame}

\begin{frame}
\frametitle{Server Client keys}
\begin{itemize}
\item xpdAcq now has a system of ``server'' and ``client'' keys to match these data
sets.

\item In this system ``clients'' and ``servers'' will have a shared key.
Thus, if one wanted to search for which calibration(s) go with a given data set
one just searches
\texttt{db(detecter\_calibration\_server\_uid=
header['detector\_calibration\_client\_uid'])}.

\item This system will allow for (in a future release) overriding calibrations and
other pieces of linked data and data analysis when calibrations are taken after
the sample data.
\end{itemize}
\end{frame}

\section{Tramp/Tlist shutter protocol}

\begin{frame}
\frametitle{Tramp/Tlist shutter protocol}
\begin{itemize}
\item Keeping the detector as clean/sensitive as possible
\item Some scans don't take images continously
(eg while waiting for temperature equilibration)
\item Temperature based scan plans now have the option to close the shutter
when not taking images (see documentation for details)
\end{itemize}
\end{frame}

\section{Data Reduction and Visualization Pipeline}

\begin{frame}
\frametitle{Data Reduction and Visualization Pipeline - Motovation}

\begin{itemize}
\item Users walk away with resonably analyzed data \\
(using some of the best tools in the community)
\item Users use XPD/NSLS-II/BNL's high powered computational resources
\item Allows for human-in-the-loop experiment tuning \\
(does my data look good?)
\end{itemize}

\end{frame}

\begin{frame}
\frametitle{Usage and capabilities}
The system is designed to do what you expect:

\begin{itemize}
\item Visualizes data as it comes off the detector
    \begin{itemize}
    \item dark subtracted image
    \item mask
    \item $I(Q)$
    \item $I(2\theta)$
    \item $F(Q)$
    \item $G(r)$
    \end{itemize}
\item Writes files to \texttt{tiff\_base} as data is analyzed
    \begin{itemize}
    \item dark subtracted image
    \item calibration \texttt{.poni} file
    \item mask \texttt{.msk} file
    \item $I(Q)$
    \item $I(2\theta)$
    \item $G(r)$
    \item metadata
    \end{itemize}
\end{itemize}
\end{frame}

\begin{frame}
\frametitle{Live Demo}
\end{frame}

\section{Conclusion}
\begin{frame}
\frametitle{Conclusion}
Much of this release is infrastructural, some of which is easy to see
(live visualization), others less so (server client links).

Long term these will help us to accive many of our goals:
\begin{itemize}
\item database of analyzed data
\item adaptive scan logic
\item automated experimentation
\end{itemize}
\end{frame}

\end{document}
